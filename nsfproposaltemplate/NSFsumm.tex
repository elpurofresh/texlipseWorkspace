% !TEX root = NSFmaster.tex
%%%%%%%%% SUMMARY -- 1 page, third person
\begin{center}
  {\large \bf RI: Small: Improving State Estimation using Aggressive Maneuvers for Aerial Robots}
\end{center}
\begin{center}
\section*{Project Summary}
\end{center}

Unmanned Aerial Vehicles (UAVs) are indispensable for various applications where human intervention is impossible, risky or expensive e.g. hazardous material recovery, traffic monitoring, disaster relief support, military operations etc.  A key issue for UAVs is their inability to operate during periods  where GPS coverage is transient or unavailable. Contrary to remote-controlled tasks in a high altitude, low-altitude flight in an urban environment requires a higher level of autonomy to respond to complex and unpredictable situations.  There is a strong motivation, therefore, to develop  algorithms that estimate the state(i.e., position, velocity and attitude) in gps-denied environments for use on these UAV platforms. Combining vision/LIDAR with inertial sensors in an extended kalman filter framework has been a promising approach in these situations because of the complementary nature of the sensing modalities.  Such estimators have to cope with fundamental limits of feature tracking: If enough features are not detected and tracked the accuracy of the state estimator deteriorates. In this proposal we plan to address these issues by designing flight trajectories for accurate state estimation. We propose to a) Design specific aerial maneuvers that specifically consider the dynamics of the aerial vehicle to increase the accuracy of the state estimation. Specifically we propose to design flight trajectories that use an augmented information matrix as the cost function for an non-linear optimal control problem. The information matrix is obtained by using an extended information filter that increases the quality of the features tracked. When the accuracy of the state estimator falls below a certain threshold the aerial vehicle executes these trajectories to improve its  pose estimate; b) Develop control algorithms for vehicle path planning that maximize the information content of the sensor measurements and are dynamically feasible i.e., the flight path of the vehicle is optimized for detecting the maximum number of features. Current algorithms for estimation are decoupled from control and navigation. If the accuracy of the estimator decreases the controller becomes conservative. Our proposal departs from conventional wisdom: we specifically execute maneuvers to increase estimator accuracy. An integral part of this proposed work is the validation of these algorithms in outdoor environments using an autonomous helicopter and a fixed wing aerial vehicle. \\
\noindent \textbf{Intellectual Merit:}  The proposed work develops algorithms for autonomous navigation of aerial robots in gps-denied environments. The proposed work will develop and experimentally validate algorithms that will lay the foundation for introduction of unmanned aerial systems into the civilian airspace. There are two technical challenges addressed in this proposal : i) Development of aerial maneuvers that increase the accuracy of state estimates; ii) Designing flight paths that maximize the sensing information. Algorithms are proposed to solve both of them. At a higher level, the intellectual merit of this work lies in the exploration of synergy between control and estimation.\\
\noindent \textbf{Broader Impacts:} The broader impacts of the proposed research activity include: (i) the development of new tools for state estimation that go beyond the aerial robot application (e.g., navigation in GPS-denied environments, underground mines, underwater etc  for ground and underwater robots);  ii) development of a year long senior capstone project on aerial robotics for senior undergraduate students in the School of Earth and Space Exploration (SESE); iii)The development of robotics resources in terms of K-12 education at SESE and making them an integrated part of the Mars Education Program. 