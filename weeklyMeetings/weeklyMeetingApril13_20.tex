\documentclass[a4paper, 10pt]{article}
%\documentclass[a4paper, 10pt, article]{ieeeconf}
\usepackage{graphicx}

\begin{document}

\title{Weekly Meeting, April 13th-20th, 2011
\\Multiple Mobile Social Robots}
\author{Andres Mora}
\date{} % to elminate the date

\maketitle

\section{Issues}
\begin{enumerate}
  \item Initial simulator's model
  \item SCM's journal strategy
\end{enumerate}

\section{Simulator model}
We can identify two different operator ``modes'': monitoring and teleroperating.
The first one can be modelled as the cost of neglecting the error's of all of the robots in the system.
The second one can be modelled in connection with the first one, and it should represent the cost of not acting upon a robot's identified error state.

\subsection{Monitoring}
It is represented by a utility function that identifies if a robot is in an error state.
This should represent how badly the system is operating.
The operator should decrease this badness for all the robots. 
No ``badness'' is obtained if every robot's error is identified.

Some of the possible parameters for this ``cost function'' should include among others:
\begin{itemize}
  \item environmental conditions
  \item operator's expertise (in terms on how skilled s/he is).
  \item penalty for not identifying errors
  \item others\ldots
\end{itemize}


\section{Journal strategy}
Use the submitted IROS as the base for the journal. 
Add the set of requirements that are characteristic for the design of teleoperation systems within the scope of mobile social robotics, particularly their GUI.
Add the observations made during the experiments of last year and reported on the HRI paper.


\end{document}